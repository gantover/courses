\documentclass{report_template}
\usepackage{blindtext}

\author{Damian Hubert}
\title{Linear Algebra}
\date{\today}

%%% Main Document Space %%%
\begin{document}

\maketitle
\tableofcontents

\chapter{Vectors}

\begin{multicols*}{2}

\section{Algebraic and geometric Interpretation of vectors}

\begin{itemize}
  \item A vector is an \textbf{ordered list} of numbers 
  \item Dimensionality = \# elements 
  \item Vectors can be displayed in \textbf{column} or \textbf{row} 
  \item Common notations : \textbf{v}, $\vec{v}$ 
  \item \textbf{Geometric Vector} have lenght, direction properties, \textbf{the starting point does not matter} 
  \item The starting point is called \textit{tail} and the end point \textit{head} 
  \item Starting at the \textbf{origin} is called \textbf{\textit{Standard position}} 
  \item Vectors can also \textbf{contain functions} $\begin{pmatrix} \sin(x)\\ x\cos(x) \end{pmatrix}$ 
\end{itemize}


\section{Vector addition and substraction}

\begin{itemize}
  \item The dimensionality has to be the same for algebraic vector operations 
  \item Geometric addition : put one vector tail onto the previous vector head 
  \item Substraction $\equiv$ addition of -$\vec{v}$ 
  \item Vector substraction as "head to head"
\end{itemize}

\end{multicols*}


\end{document}
